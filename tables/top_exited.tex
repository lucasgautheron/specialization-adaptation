\begin{table}[H]
\centering
\caption{Physicists with the highest change scores $c_a$. $D(\bm{I_a})$ and $D(\bm{S_a})$ measure the diversity of intellectual and social capital. Numbers in parentheses indicate the share of attention dedicated to each research area during each time-period. Asterisks ($\ast$) indicate physicists with a permanent position.}
\label{table:top_exited}
\begin{tabular}{p{0.15\textwidth}|c|c|c|b{0.25\textwidth}|b{0.25\textwidth}}
\toprule
             Physicist & $c_a$ & $D(\bm{I_a})$ & $D(\bm{S_a})$ &                            Previous main area &                                Current main area \\
\midrule
 W.Buchmuller ($\ast$) &     1 &          8.09 &         10.87 &  Neutrinos \& flavour physics (0.26$\to$0.06) & String theory \& supergravity (0.12$\to$0.36)\\ \hline
   P.M.Lavrov ($\ast$) &     1 &          3.82 &         10.78 &                   Quantum Field Theory (0.58) &                   Quantum Field Theory (0.43)\\ \hline
    G.Mangano ($\ast$) &     1 &          8.56 &         10.72 &           Neutrinos \& flavour physics (0.32) &           Neutrinos \& flavour physics (0.46)\\ \hline
      Y.Satoh ($\ast$) &     1 &          7.62 &          5.74 &                       AdS/CFT (0.23$\to$0.10) & String theory \& supergravity (0.20$\to$0.42)\\ \hline
             M.Zabzine &     1 &          5.11 &          7.10 &          String theory \& supergravity (0.53) &          String theory \& supergravity (0.68)\\ \hline
              M.Visser &     1 &          9.06 &          7.97 &                            Black holes (0.22) &                            Black holes (0.38)\\ \hline
   T.R.Morris ($\ast$) &     1 &          7.48 &         11.61 &          Perturbative methods (0.35$\to$0.05) &                       AdS/CFT (0.07$\to$0.27)\\ \hline
  Subir Ghosh ($\ast$) &     1 &          6.88 &          8.18 &                   Quantum Field Theory (0.43) &                   Quantum Field Theory (0.21)\\ \hline
      Z.Lalak ($\ast$) &     1 &          7.31 &          9.95 &                              Cosmology (0.32) &                              Cosmology (0.26)\\ \hline
           A.Falkowski &     1 &          7.93 &         10.92 &                     Electroweak sector (0.37) &                     Electroweak sector (0.36)\\ \hline
         Jorge G.Russo &     1 &          9.19 &          9.20 &          String theory \& supergravity (0.38) &          String theory \& supergravity (0.30)\\ \hline
 T.A.Jacobson ($\ast$) &     1 &          9.71 &          9.97 &                            Black holes (0.34) &                            Black holes (0.49)\\ \hline
            V.Niarchos &     1 &          5.37 &          7.77 &          String theory \& supergravity (0.44) &          String theory \& supergravity (0.56)\\ \hline
           H.Samtleben &     1 &          5.65 &          7.50 &          String theory \& supergravity (0.52) &          String theory \& supergravity (0.67)\\ \hline
    H.Nicolai ($\ast$) &     1 &          8.15 &          8.80 & String theory \& supergravity (0.36$\to$0.14) &                   Dark matter (0.00$\to$0.17)\\ \hline
     C.T.Hill ($\ast$) &     1 &          9.09 &         12.16 &            Electroweak sector (0.22$\to$0.16) &                           QCD (0.11$\to$0.16)\\ \hline
           O.Corradini &     1 &         10.03 &         10.98 &                     Cosmology (0.18$\to$0.00) &          Quantum Field Theory (0.12$\to$0.41)\\ \hline
F.Bastianelli ($\ast$) &     1 &          8.39 &          9.55 &                   Quantum Field Theory (0.25) &                   Quantum Field Theory (0.31)\\ \hline
     Malcolm Fairbairn &     1 &          8.32 &         10.65 &                            Dark matter (0.36) &                            Dark matter (0.38)\\ \hline
          A.Tomasiello &     1 &          2.98 &          5.54 &          String theory \& supergravity (0.74) &          String theory \& supergravity (0.60)\\ \hline
\bottomrule
\end{tabular}
\end{table}
