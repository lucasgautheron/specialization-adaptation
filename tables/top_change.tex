\begin{table}
\centering
\caption{Physicists with the highest change scores $c_a$. $D(\bm{I_a})$ and $D(\bm{S_a})$ measure the diversity of intellectual and social capital. Numbers in parentheses indicate the share of attention dedicated to each research area during each time-period. Asterisks ($\ast$) indicate physicists with a stable affiliation (tenured positions).}
\label{table:top_change}
\begin{tabular}{p{0.15\textwidth}|c|c|c|b{0.25\textwidth}|b{0.25\textwidth}}
\toprule
              Physicist & $c_a$ & $D(\bm{I_a})$ & $D(\bm{S_a})$ &                             Previous main area &                                Current main area \\
\midrule
                T.Fuchs &  0.93 &          4.38 &          6.50 &                            QCD (0.53$\to$0.00) & Neutrinos and flavour physics (0.00$\to$0.64)\\ \hline
   N.G.Sanchez ($\ast$) &  0.88 &          7.52 &         13.08 &                    Black holes (0.37$\to$0.03) &                   Dark matter (0.00$\to$0.57)\\ \hline
   V.V.Braguta ($\ast$) &  0.88 &          5.03 &          8.04 &                        Hadrons (0.45$\to$0.00) &                           QCD (0.05$\to$0.60)\\ \hline
            G.B.Cleaver &  0.83 &          6.34 &         12.22 & String theory and supergravity (0.40$\to$0.00) &                     Cosmology (0.03$\to$0.62)\\ \hline
      J.Martin Camalich &  0.82 &          2.34 &          4.57 &                            QCD (0.78$\to$0.03) &                       Hadrons (0.06$\to$0.38)\\ \hline
Kwei Chou Yang ($\ast$) &  0.81 &          3.58 &          7.80 &                        Hadrons (0.63$\to$0.03) &                   Dark matter (0.00$\to$0.78)\\ \hline
      S.D.Katz ($\ast$) &  0.80 &          6.50 &         10.85 &                    Dark matter (0.42$\to$0.03) &                           QCD (0.06$\to$0.47)\\ \hline
    G.Ferretti ($\ast$) &  0.78 &          5.22 &          6.18 & String theory and supergravity (0.52$\to$0.00) &            Electroweak sector (0.01$\to$0.58)\\ \hline
         K.Bhattacharya &  0.78 &         10.79 &         10.59 &  Neutrinos and flavour physics (0.20$\to$0.00) &                     Cosmology (0.09$\to$0.37)\\ \hline
              H.Ebrahim &  0.78 &          3.90 &          6.81 &                    Black holes (0.44$\to$0.06) &                       AdS/CFT (0.02$\to$0.34)\\ \hline
              K.Narayan &  0.77 &          3.83 &          7.13 & String theory and supergravity (0.62$\to$0.00) &                       AdS/CFT (0.03$\to$0.52)\\ \hline
       Mario Campanelli &  0.77 &          2.04 &          4.48 &  Neutrinos and flavour physics (0.79$\to$0.06) &            Electroweak sector (0.09$\to$0.64)\\ \hline
          V.K.Oikonomou &  0.76 &          8.05 &          6.18 &                 Thermodynamics (0.24$\to$0.02) &                     Cosmology (0.00$\to$0.47)\\ \hline
               K.Sasaki &  0.76 &          3.14 &          8.87 &               Collider physics (0.68$\to$0.06) &                Thermodynamics (0.00$\to$0.27)\\ \hline
  H.L.Verlinde ($\ast$) &  0.74 &          5.32 &          8.88 & String theory and supergravity (0.60$\to$0.04) &                       AdS/CFT (0.02$\to$0.35)\\ \hline
                 Bin Wu &  0.74 &          5.02 &          8.01 &                            QCD (0.41$\to$0.02) &              Collider physics (0.15$\to$0.56)\\ \hline
            Eiji Nakano &  0.74 &          6.90 &         10.13 &                            QCD (0.39$\to$0.06) &                       Hadrons (0.04$\to$0.78)\\ \hline
    A.T.Suzuki ($\ast$) &  0.73 &          4.36 &          6.43 &           Perturbative methods (0.61$\to$0.04) & Neutrinos and flavour physics (0.01$\to$0.39)\\ \hline
             A.Mariotti &  0.73 &          3.81 &          4.91 & String theory and supergravity (0.63$\to$0.01) &            Electroweak sector (0.01$\to$0.38)\\ \hline
              I.Scimemi &  0.73 &          6.77 &          8.60 &                        Hadrons (0.36$\to$0.00) &              Collider physics (0.09$\to$0.81)\\ \hline
\bottomrule
\end{tabular}
\end{table}
